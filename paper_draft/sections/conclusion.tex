\section{Conclusion}
\label{sec:conclusion}

In this work, we investigated the stability of Large Language Model (LLM) alignment in extended multi-turn conversations. We hypothesized that aligned models would regress to their base-level priors---specifically, the ``helpful assistant'' persona---as conversation length increases, overriding session-specific instructions. Through a 50-turn simulated dialogue with 	\texttt{gpt-4o-mini}, enforcing a strict linguistic constraint and a distinct pirate persona, we found strong evidence supporting this hypothesis. Our results showed a significant positive correlation ($r=0.568$) between turn count and constraint violations, alongside a steady increase in regression to default assistant behavior ($r=0.446$) and a decline in persona fidelity ($r=-0.300$). These findings quantify ``alignment decay,'' suggesting that current alignment techniques produce fragile behavioral states that erode as the context window fills with model-generated content.

Future work should explore mitigation strategies, such as periodic reminder injections or dynamic prompt management, to maintain alignment over extended interactions. Additionally, investigating whether different model architectures or larger models exhibit similar decay patterns could provide deeper insights into the underlying mechanisms of alignment stability. Ultimately, developing robust methods for long-context alignment is crucial for enabling the next generation of persistent AI agents and personalized assistants.
